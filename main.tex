\documentclass{article}

% Language setting
% Replace `english' with e.g. `spanish' to change the document language
\usepackage[english]{babel}

% Set page size and margins
% Replace `letterpaper' with `a4paper' for UK/EU standard size
\usepackage[letterpaper,top=2cm,bottom=2cm,left=3cm,right=3cm,marginparwidth=1.75cm]{geometry}

% Useful packages
\usepackage{amsmath}
\usepackage{graphicx}
\usepackage[colorlinks=true, allcolors=blue]{hyperref}

\title{Analyzing Frequent Changes in Open Source Codebases and Their Causes}

\begin{document}
\maketitle

\section{Related Work}

Understanding the mechanisms and factors influencing software changes is crucial for enhancing software maintenance and evolution. Early research in this area focused on categorizing software changes to better understand the nature of modifications that occur over the life of a software project. Notably, Swanson \cite{swanson1976dimensions} classification of maintenance types—corrective, adaptive, and perfective—provided a seminal framework for categorizing software changes. This framework has been instrumental in subsequent research, guiding the analysis of how different types of changes affect software systems.\\

Building on Swanson's work, numerous studies have further refined the classification of software changes. For instance, Mockus \cite{mockus2000identifying} introduced a keyword-based approach to classify changes within large software systems. This work emphasized the importance of distinguishing between corrective, adaptive, and perfective changes, finding that different types of changes tend to exhibit distinct characteristics, such as size and complexity.\\


\subsection{Code Change Patterns and Their Implications}

A significant body of research has focused on identifying and analyzing patterns in code changes to better manage software evolution. These patterns provide insights into the implications of different types of modifications, such as feature additions, bug fixes, and code refactoring. For instance, Buckley\cite{buckley2005towards} explored the relationship between software evolution and dependency management, underscoring the complexity introduced by certain types of changes. Similarly, Mens\cite{mens2003towards} provided a taxonomy of software evolution, advocating for a structured understanding of changes to improve software maintenance.

Further refining the understanding of code change patterns, Trautsch \cite{trautsch2023really} conducted an extensive analysis of quality-improving commits, discovering that perfective changes often reduce complexity, while corrective changes may increase both the size and complexity of the code. This distinction is crucial for assessing whether files with high change frequency are more associated with bug fixes or feature additions, offering a lens through which to evaluate the impact of these changes on software maintainability.

Additionally, Palomba \cite{palomba2017exploratory} explored the impact of frequent changes on code maintainability, finding that classes often subject to bug fixes are subsequently improved in terms of maintainability, whereas frequent feature additions tend to lead to refactoring due to issues like code duplication and technical debt. Jaafar \cite{jaafar2017analyzing} introduced the concept of Asynchrony change patterns, focusing on sets of files that consistently change together, which is independent of the developers involved, providing insights into the coordination required in managing high-change areas of the codebase.

These studies collectively underscore the importance of understanding the reasons behind frequent changes, particularly in the context of maintaining high software quality. \\ 

\subsection{Impact of Bug Fixes and Repetitive Changes}
A more specific area of research within the realm of software change analysis focuses on the impact of bug fixes and the repetitive nature of certain changes. Ufuktepe\cite{ufuktepe2021relation} examined common bug fix patterns, identifying modifications to method bodies as a frequent change with typically low overall impact on software structure. This suggests that certain files or lines may repeatedly undergo similar changes, pointing to potential underlying issues that necessitate frequent fixes.\\

Nguyen \cite{nguyen2013study} further investigated the repetitiveness of changes, finding that small changes often exhibit high repetitiveness, while larger changes do not. This insight is particularly relevant for identifying files and lines in version control histories that exhibit an extraordinary frequency of changes or bug fixes. Understanding these repetitive patterns is essential for developing strategies to mitigate their occurrence.\\

Misirli \cite{misirli2016studying} specifically examined high-impact fix-inducing changes (HIFCs) in software systems. This study introduced a measure of impact based on churn, number of files, and subsystems modified by subsequent fixes. The authors found that lines of code added, number of developers involved, and prior modifications on files are strong indicators of HIFCs. This study highlights the importance of prioritizing high-impact changes, which is relevant for understanding and addressing files with extraordinary numbers of changes or bug fixes. 

\subsection{Tools and Techniques for Change Detection}
The detection and analysis of software changes have been greatly facilitated by the development of various tools and techniques. Version control systems, along with automated tools, play a crucial role in tracking changes and understanding their impacts. For example, Fluri et al. \cite{fluri2007change} developed \texttt{ChangeDistiller}, an Eclipse plug-in that uses a tree differencing algorithm to extract fine-grained source code changes. This tool has been foundational in automating the analysis of software changes.

Further advancements in tooling, such as PYCT by Lin \cite{lin2016empirical}, have extended the capabilities of change detection, reducing the manual effort required for change extraction and classification. Additionally, Weissgerber \cite{weissgerber2006identifying} introduced methods for ranking code changes as refactoring changes through signature-based analysis and clone detection. These tools and techniques are invaluable for our research, as they provide the means to identify and analyze files and lines with an extraordinarily high frequency of changes or bug fixes.
\\

Despite the extensive body of research on software change analysis, there remains a significant gap in understanding the specific reasons behind files or lines in open-source projects exhibiting an extraordinarily large number of changes or bug fixes. While previous studies have categorized changes and explored their impact on software quality, the precise causes driving high-frequency changes in specific parts of the codebase have not been fully explored.\\

Our research seeks to fill this gap by focusing on the following objectives:

\begin{itemize}

\item{\textbf{Identifying Files with High Change Frequency}: Using version control histories from popular open-source projects, we will identify files and lines with an extraordinary number of changes or bug fixes.}

\item{\textbf{Creating a Taxonomy of Change Reasons}: We will develop a taxonomy categorizing the reasons behind these frequent changes, drawing on both automated analysis and manual inspection.}

\item{\textbf{Offering Mitigation Strategies}: Based on the identified reasons, we will discuss potential strategies for reducing the frequency of these changes, aiming to improve software maintainability and quality.}
\end{itemize}

\bibliographystyle{ieeetr}
\bibliography{references}

\end{document}
